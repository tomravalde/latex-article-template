\section{Model development} \label{sec:models}
The aim of this section is to show the first steps in the development of an integrated resource management model. \citet{Samsatli} is used as a starting point because its general nature makes it simple to extend resources and processes which concern energy to a broader scope which includes water and waste resources and processes such as agriculture, livestock and household appliances. As specified in Section~\ref{sec:aims}, an initial model will be a prototype based on a small subsistence community, with no spatial or temporal disaggregation. There are two reasons for this: 
\begin{enumerate}
	\item A system of this type is simple to formulate, before moving onto larger, more complicated networks.
	\item Even subsistence communities are over-consuming resources \citep{Hobbes2005, Hobbes2007, Schandl2006, Haberl2001b}. Thus, the integrated resource model formulated here can have a useful real-world application.
\end{enumerate}

Having reviewed metabolism studies and material flow analyses for a number of subsistence communities, it was decided that Tat Hamlet (Vietnam) would make a suitable case study because of the availability of both a materials flow analysis \citep{Hobbes2007, Schandl2006} and an energy flow analysis \citep{Heezen2004}. Tat is a village of 105 households and 69 hectares of farmland. The MFA studies don't comprehensively list all possible resource flows, therefore other similar studies of other areas have been consulted to assist the creation of a plausible network of resources and processes\footnote{Examples include \citet{Alam1997}, \citet{Alam1999} and \citet{Tripathi2001}.}. Thus, a realistic resource-process network can be formulated from known data about Tat and processes and resource flows observed in similar contexts. 

\subsection{Tat Hamlet resource flow model}
The first stage of building the model is to decide on the resources and processes that will be modelled. The resources considered are given in Table~\ref{tab:tat_resources}. The processes in the village which consume and produce these resources are:
\begin{itemize}
	\item agriculture,
	\item aquaculture,
	\item cooking,
	\item household appliances,
	\item household heat,
	\item livestock,
	\item transport.
\end{itemize}

\begin{figure}[h]
	\centering
	% Author: Till Tantau
% Source: The PGF/TikZ manual

\begin{tikzpicture}[->,>=stealth',shorten >=1pt,auto,node distance=1.4cm,
                    semithick]
  \tikzstyle{every state}=[draw]
  \tikzstyle{task}=[rectangle, draw, minimum width=3.3cm]
% Processes
  \node[task] (agr) {Agriculture};
  \node[task] (aqua) [above of=agr] {Aquaculture};
  \node[task] (cook) [below of=agr] {Cooking};
  \node[task] (heat) [below of=cook] {Heating};
  \node[task] (apps) [below of=heat] {Household appliances};
  \node[task] (LS) [below of=apps] {Livestock};
  \node[task] (trans) [below of=LS] {Transport};
% Resources (no demand)
  \node[state] (Wst) [left of=aqua, node distance=4.8cm] {Wst};
  \node[state] (B) [left of=agr, node distance=4.8cm] {B};
  \node[state] (WH) [left of=cook, node distance=4.8cm] {WH};
  \node[state] (E) [left of=heat, node distance=4.8cm] {E};
  \node[state] (P) [left of=apps, node distance=4.8cm] {P};
  \node[state] (M) [left of=LS, node distance=4.8cm] {M};
% Resources (demand)
  \node[state] (Wtr) [right of=aqua, node distance=4.8cm] {Wtr};
  \node[state] (F) [right of=agr, node distance=4.8cm] {F};
  \node[state] (H) [right of=cook, node distance=4.8cm] {H};
  \node[state] (L) [right of=heat, node distance=4.8cm] {L};
% Dummy nodes to draw demand arrows
  \node[] (Wtr-D) [right of=Wtr] {};
  \node[] (F-D) [right of=F] {};
  \node[] (H-D) [right of=H] {};
  \node[] (L-D) [right of=L] {};
% Dummy nodes for import arrows
  \node[] (B-I) [left of=B]  {};
  \node[] (E-I) [left of=E]  {};
  \node[] (P-I) [left of=P]  {};
  \node[] (Wtr-I) [above of=Wtr]  {};
  \node[] (F-I) [above right of=F]  {};
% Paths for resources (no demand)
  \path (agr) edge (WH);
  \path (apps) edge (WH);
  \path (cook) edge (Wst);
  \path (heat) edge (Wst);
  \path (B) edge (aqua);
  \path (B) edge (agr);
  \path (B) edge (cook);
  \path (B) edge (heat);
  \path (aqua) edge (B);
  \path (agr) edge (B);
  \path (E) edge (agr);
  \path (E) edge (apps);
  \path (P) edge (trans);
  \path (P) edge (agr);
  \path (M) edge (agr);
  \path (LS) edge (M);
% Paths for resources (demand)
  \path (Wtr) edge (agr);
  \path (agr) edge (F);
  \path (aqua) edge (F);
  \path (cook) edge (H);
  \path (heat) edge (H);
  \path (apps) edge (L);
% Demand arrows
  \draw[dashed,->] (Wtr) -- (Wtr-D);
  \draw[dashed,->] (F) -- (F-D);
  \draw[dashed,->] (H) -- (H-D);
  \draw[dashed,->] (L) -- (L-D);
% Import arrows
  \draw[->>] (Wtr-I) -- (Wtr);
  \draw[->>] (B-I) -- (B);
  \draw[->>] (E-I) -- (E);
  \draw[->>] (P-I) -- (P);
  \draw[->>] (F-I) -- (F); 
\end{tikzpicture}

	\caption{Resource-technology network in Tat. Cirles represent resources (abbreviations given in Table~\ref{tab:tat_resources}); boxes represent process, dotted arrows represent demands; and double headed arrows represent imports.} \label{fig:tat_network}
\end{figure}

	\begin{table}[h]
		\centering
		\caption{Resource details for Tat Hamlet resource flow model.} \label{tab:tat_resources}
			\begin{tabular}{lllll}
			\toprule
			Resource & Abbreviation & Cost [cents/kg or cents/MJ] & \multicolumn{2}{l}{Demand} \\
			\midrule
			Biomass & B & 0.72 & & \multirow{6}{*}{[kg/capita/year]} \\
			Food & F & 200 & y & \\
			Manure & M & & & \\
			Petrol & P & 104 & & \\
			Waste & Wst & & & \\
			Water & Wtr & 0.05 & 3133334 & \\
			\midrule
			Electricity & E & 1.94 & & \multirow{4}{*}{[MJ/capita/year]} \\
			Heat & H & & 10150 & \\
			Light & L & & 5.44 & \\
			Waste heat & WH & & 61.2 & \\
			\bottomrule
			\end{tabular}
	\end{table}

 

The $R$ resources and $P$ processes can be linked together in a network which defines all possible paths from source to sink (final demand and wastes) as shown in Figure~\ref{fig:tat_network}. For each process, the relationship between inputs and output resource quantities is defined with the production coefficient $k_{rp}$ which reflect the net resource balance through the process. For example, for $k_{rp}=k_{electricity, household~appliances}=-10$ and $k_{light, household~appliances}=1$, thus 1 unit of light is produced for every 10 units of electricity consumed (indicating a 10\% efficiency of the light---shown by the fact that $k_{waste~heat, household~appliances}=9$).

Tables~\ref{tab:tat_params} and \ref{tab:tat_vars} give the parameters and decision variables for the model\footnote{Where units are given in kg, they can be replaced with MJ for the case of energy resources.}. The objective function in Equation~\eqref{eq:tat_basic_OF} minimises the cost of importing resources. The primary constraint given by Equation~\eqref{eq:tat_basic_balance} is a mass balance of each resource, such that the sum of the imports and resource consumption in any process equals that of exports and resource production from processes. Constraint~\eqref{eq:tat_basic_processLimits} set boundaries on the number of any particular processes which can take place (corresponding to land amount or the number of houses in which a process is adopted). A non-negativity constraint is applied to the imports in Equation~\eqref{eq:tat_basic_imports}. Finally, a variable constraint on emissions is imposed on the system in Equation~\eqref{eq:tat_basic_emissions}---the limit of this value can be varied and plotted against the objective function value in a Pareto-optimisation of cost and emissions.
\begin{align}
	\min_{I_r,E_r,(NP)_p} \Bigg\{z&=\sum_{r=1}^R c_rI_{r} \Bigg\} \label{eq:tat_basic_OF} \\
	I_r&=D_r-\sum_{p=1}^{P}k_{rp} P_p NP_p + E_r \label{eq:tat_basic_balance} \\
	NP_{min} &\leq NP \leq NP_{max} \label{eq:tat_basic_processLimits} \\			
	I_r &\geq 0 \label{eq:tat_basic_imports} \\
	\sum_{p=1}^P e_p P_p (NP)_p &\leq \mbox{emissions target} \label{eq:tat_basic_emissions}.
\end{align}
	\begin{table}[h]
		\centering
		\caption{Parameters in Tat resource flow optimisation model.} \label{tab:tat_params}
			\begin{tabular}{lll}
			\toprule
			$e_p$ & Emissions from a process & [kg/capita/year] \\
		        $c_r$ & Cost of resource & [cents/kg] \\		
			$D_r$ & Demand for resource & [kg/capita/year] \\
			$k_{rp}$ & Coefficient defining production/consumption ratio of resource in process & [-] \\
			$P_p$ & Rate of process operation & [kg/capita/year]\\
			\bottomrule
			\end{tabular}
	\end{table}

 
	\begin{table}[h]
		\centering
		\caption{Variables in Tat resource flow optimisation model.} \label{tab:tat_vars}
			\begin{tabular}{lll}
			\toprule
			$I_r$ & Import of resource & [kg/capita/year]\\ 
			$E_r$ & Export of resource & [kg/capita/year] \\
			$NP_p$ & Number of each process & [-] \\
			\bottomrule
			\end{tabular}
	\end{table}

 

\subsubsection*{Implementation and results}
The Python library `Pyomo'\footnote{\texttt{https://software.sandia.gov/trac/coopr/wiki/Pyomo}.} is used to build the model (which is a front end language compiler for mathematical programming similar to GAMS\footnote{\texttt{http://www.gams.com/}.}), and solved by the GNU Linear Programming Kit\footnote{\texttt{http://www.gnu.org/software/glpk/}.} performing linear programming. A Pareto-optimisation is carried out where the allowable emissions limit is varied whilst the objective function (import costs) is minimised. Optimial variable values are returned for exports, imports and process numbers. The results are presented in Figure~\ref{fig:tat_pareto} and Tables~\ref{tab:tat_basic_results} and \ref{tab:tat_basic_NP_results}.
\begin{figure}
	\centering
	\includegraphics[width=0.75\textwidth]{./img/tat_pareto.pdf}
	\caption{Pareto optimisation for emissions and resource import costs.} \label{fig:tat_pareto}
\end{figure}
\begin{table}[h]
	\centering
	\caption{Optimal import and export values for the Tat Hamlet resource flow model, for the minimum financial cost case. Units are as for Table~\ref{tab:tat_resources}.} \label{tab:tat_basic_results}
	\begin{tabular}{lll}
		\toprule
		Resource  & Imports & Exports \\
		\midrule
		Biomass & & 820 \\
		Electricity & 68 & \\
		Food & 171 & \\
		Heat & & \\
		Light & & \\
		Manure & & \\
		Petrol & 2 & \\
		Water & 4700000 & \\
		Waste heat & & 61 \\
		Waste & & 113 \\
		\bottomrule
	\end{tabular}
\end{table}

 
\begin{table}[h]
	\centering
	\caption{Optimal process number values for the Tat Hamlet resource flow model for the minimum financial cost case.} \label{tab:tat_basic_NP_results}
	\begin{tabular}{lll}
		\toprule
		Process  & NP & Comments \\
		\midrule
		Agriculture & 69 & Hectares of land \\
		Aquaculture & 100 & \% of homes with ponds \\
		Cooking & 105 & Number of homes with stoves \\
		Household Appliances & 105 & Number of homes with electric lighting \\
		Household heat & 105 & Number of homes with heating facilities \\
		Livestock & 87 & Number of homes owning livestock \\
		Transport & & \\
		\bottomrule
	\end{tabular}
\end{table}

 

\subsubsection*{Discussion}
The following results are noted:
\begin{itemize}
	\item The Pareto optimisation in Figure~\ref{fig:tat_pareto} shows that emissions and costs can be optimised whilst allocating resources and processes.
	\item The program can prompt ideas for improved resource management, for example, lots of biomass is exported (see Table~\ref{tab:tat_basic_results}). The village could buy an electricity generator which uses biomass as feedstock. This would save on having to import electricity.
	\item It is interesting that the model suggests that it isn't beneficial for the every household to own livestock. Above a certain threshold, their food demand may be such that it limits land availability for human food production (hence increasing the amount required to be imported).
\end{itemize}	
The model as it stands suffers limitations. For example, it doesn't consider technology costs or income from exports. Some resource categories are currently too homogenous to accurately represent reality (for example, `food' could be separated into food for human consumption and food for animal consumption, and waste could be separated into multiple streams). More processes and resources could be added. for example, water treatment processes, which generates both clean water and wastewater. Given the results and the limitations of the model, future work can consider the following:
\begin{itemize}
	\item The model could include a `technology choice' component (for example to test the financial savings possible with a biomass based energy generator). This would take a database of available technologies as inputs, and determine whether or not the technology should be included in the village (thus, a Boolean decision variable representing each technology should be introduced).
	\item Future developments can make the model more sophisticated to overcome the limitations suggested in this section.
	\item Finanlly, resource quality consideration can be taken into account. This is the subject of the next section.
\end{itemize}

\subsection{Considering quality considerations in a water network}
The resource allocation model described above does a good job of allocating resources to users in the village, however it doesn't yet consider the quality of those resources. For example, water for human consumption (in final demand) must be achieve a minimum standard of quality. Moreover, the quality of resources out of any process will depend on the quality of resources going into the process. Thus a method is required which can account for this. Two approaches will now be described.

\subsubsection*{The `null model' approach}
There is a simple way to model resource qualities, such that no significant changes are required to be made to the first model---this shall be termed the `null model'. In this approach each resource is sub-divided into several discrete resources of differing quality (for example, there could be five different water qualities: very poor, poor, average, good and very good, which are represented by five different resources). Poor water can go through a treatment process to improve quality (which would have an associated energetic, and hence financial cost). Final demands can take any water resource with a quality above a minimum threshold. Clearly a model of this nature will become rapidly become very large and inelegent for networks of significant magnitude, presenting problems of computational tractability. Furthermore, the discrete levels of quality fail to capture the continous resource quality exhibited in real systems.  Thus, a key task of this research is to model the resources and their qualities in a computationally tractable fashion which sufficiently represents real situations. The resulting model can then be compared to the null model for size and speed.

\subsubsection*{Quality model development (James Keirstead)}
In idea to incorporate quality considerations has been developed by James Keirstead. The model is similar to the initial approach. However, an additional index $q$ is used to define resource qualities; and extra parameters (Table~\ref{tab:water_params}) and variables (Table~\ref{tab:water_vars}) are introduced such that each resource can be described by both quantities and qualities; and processes have a production rate for both quantity and quality. For example, $k_{rp}$ in the previous model becomes $\mbox{prod\_coeff}_{prq}$ because each process produces and consumers resources of given qualities.
	\begin{table}[h]
		\centering
		\caption{Parameters in water quality optimisation model.} \label{tab:water_params}
			\begin{tabular}{lp{7cm}}
			\toprule
			$\mbox{r\_qual}_{rq}$ & Reference unit of resource qualities (e.g. J, kg, head etc.) \\
			$\mbox{prod\_coeff}_{prq}$ & Coefficient defining production/consumption ratio of resource in a process \\
			$\mbox{D\_qual}_{rq}$ & Resource quality demanded by user \\
			$\mbox{D\_qty}_r$ & Resource quantity demanded by user \\
			$c_r$ & Unit cost of resource of given quality \\
			$\mbox{qual\_coeff}_{prq}$ & Boolean indicating if we care about the quality of a resource for any process \\
			$\mbox{cheat}_{rq}$ & \\
			\bottomrule
		\end{tabular}
	\end{table}
 
	\begin{table}[h]
		\centering
		\caption{Variables in water quality optimisation model.} \label{tab:water_vars}
			\begin{tabular}{lp{7cm}}
			\toprule
			$I_r$ & Resource imports \\
			$\mbox{prod\_rate}_p$ & Process rate \\
			$\mbox{prod}_{prq}$ & Resource qualities produced in processes \\
			$\mbox{prod\_qty}_{pr}$ & Resource quantities produced in processes \\
			$E_r$ & Resource exports \\
			\bottomrule
		\end{tabular}
	\end{table}
 

The objective function in Equation~\eqref{eq:water_OF} minimises the total cost of importing resources of required qualities. Constraints are imposed on production rates (Equations~\eqref{eq:qty_rate} and \eqref{eq:qual_rate}) and mass balances (Equations~\eqref{eq:qual_balance} and \eqref{eq:qty_balance}) for both quantity and quality of resources.

\begin{align}
	\min_{I_r,\mbox{p\_rate}_p,\mbox{p}_{prq},\mbox{p\_qty}_{pr},E_r} \Bigg\{z&=\sum_{q=1}^Q \sum_{r=1}^R c_{rq}\mbox{r\_qual}_{rq}I_{r}\Bigg\} \label{eq:water_OF} \\
	\mbox{prod\_qty}_{pr} \mbox{r\_qual}_{rq} &= \mbox{prod}_{prq} \mbox{cheat}_{rq} \label{eq:qty_rate} \\ 
	\mbox{prod}_{prq} &= \mbox{prod\_rate}_{p} \mbox{prod\_coeff}_{prq}, \mbox{    if } \mbox{qual\_coeff}_{prq} \neq 1 \label{eq:qual_rate} \\
	0 &= \sum_{r=1}^R \Bigg(\mbox{r\_qual}_{rq} I_r + \sum_{p=1}^P \mbox{prod}_{prq} - \mbox{D\_qual}_{rq} - \mbox{r\_qual}_{rq} E_r \Bigg) \label{eq:qual_balance} \\
	0 &= I_r + \sum_{p=1}^P\mbox{prod\_qty}_{pr} - \mbox{D\_qty}_{r}-E_r \label{eq:qty_balance} 
\end{align}

An initial formulation has modelled a pump (a process) which requires water to generate electricity (resources), by considering both water's mass energy (head) and the electrical energy produced as resources---see Figure~\ref{fig:pump}). The next step is to extend this formulation to consider other processes which use water, before putting it into the Tat model.

\begin{figure}[h]
	\centering
	% Author: Till Tantau
% Source: The PGF/TikZ manual

\begin{tikzpicture}[->,>=stealth',shorten >=1pt,auto,node distance=2.8cm,
                    semithick]
  \tikzstyle{every state}=[draw]
  \tikzstyle{task}=[rectangle, draw, minimum width=3.3cm]

  \node[task] (pump) {Pump};

  \node[state] (SW) [above left of=pump] {SW};
  \node[state] (E) [above right of=pump] {E};
  \node[state] (PW) [below of=pump] {PW};

  \node[] (PW-D) [below of=PW] {};
  
  \node[] (SW-i) [above of=SW] {};
  \node[] (E-i) [above of=E] {};

  \path (SW) edge (pump);
  \path (E) edge (pump);
  \path (pump) edge (PW);

  \draw[dashed,->] (PW) -- (PW-D);

  \draw[->>] (SW-i) -- (SW);
  \draw[->>] (E-i) -- (E);
\end{tikzpicture}

	\caption{Pump production model. Cirles represent resources; boxes represent process, dotted arrows represent demands; and double headed arrows represent imports.} \label{fig:pump}
\end{figure}
Running the model gives the results (Tables~\ref{tab:pump_imports_exports} and \ref{tab:pump_resource_quality}) expected (assuming 100\% efficiency): for every kilogram of water through pump, the 98.1 J of electricity gives the water 98.1 J of energy (equivalent to 1 m of head). From this point, the next stage is to expand this model to a bigger water network which considers more processes, resources and qualities (for example water treatment processes which consider water contamination as a quality indicator).

\begin{table}[h]
	\centering
	\caption{Imports and exports of resources for the pump model.} \label{tab:pump_imports_exports}
	\begin{tabular}{lll}
		\toprule
		Resource  & Imports & Exports \\
		\midrule
		Source water & 1 & \\
		Pumped water &  & \\
		Electricity & 98.1 & \\
		\bottomrule
	\end{tabular}
\end{table}

 
\begin{table}[h]
	\centering
	\caption{Production of resource qualities ($\mbox{prod}_{prq}$) for the pump model.} \label{tab:pump_resource_quality}
	\begin{tabular}{lll}
		\toprule
		Resource  & Energy & Mass \\
		\midrule
		Source water &  & -1 \\
		Pumped water & 98.1 & 1 \\
		Electricity & -98.1 & \\
		\bottomrule
	\end{tabular}
\end{table}

 
\begin{sidewaystable}[htbp]
	\centering
	\caption{Resource qualities to be considered in each process in Tat, including biological oxygen demand (BOD), chlorine content (Cl), energy (E), mass (M) and water content (WC).} \label{tab:quality_catalogue}
	\begin{tabular}{lllllllllll}
		\toprule
		& Biomass & Electricity & Food & Heat & Light & Manure & Petrol & Waste & Waste heat & Water \\
		\midrule
		Agriculture & WC, M & E & M & & & M & & M & E & M, BOD, Cl \\
		Aquaculture & & & M & & & & & & & \\
		Drying & WC, M & E &  & E & & & & & & \\
		Cooking & WC, M & & & E & & & & M & E & M, BOD, Cl \\
		Appliances & & E & & & E & & & & E & \\
		Heat & WC, M & & & E & & & & M & & \\
		Hydroelectricity & & E & & & & & & & & M, E \\
		Livestock & & & M & & & M & & & & M, BOD, Cl \\
		Pumping & & E & & & & & & & E & M, E \\
		Petrol & & & & & & & M & & & \\
		Water treatment & & E & & & & & & M & & M, BOD, Cl \\
		\bottomrule
	\end{tabular}
\end{sidewaystable}
 
