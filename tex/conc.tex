\section{Conclusions and further work}
\label{sec:conc}

\subsection{Conclusions}
Global urban trends have been examined and it has been found that the world is rapidly urbanising---especially in developing countries. Furthermore, it has been suggested that this urbanisation brings economic opportunity to the residents of towns and cities, thus any approach that seeks to mitigate against the environmental and economic challenges of urban areas must do so without opposing urbanisation itself. It is widely viewed that improved resource efficiency through integrated resource management is the best way to go about tackling these problems. Despite common agreement here, there is surprisingly little in the way of tools for optimising the integrated provision of energy, water and waste resources, even though they exhibit numerous interconnections and relationships within the urban environment. A review of optimised provision models within these fields has been undertaken with a view to developing an integrated resource model in this study. The first steps in developing a model have optimised resource flows in a small village (albeit with significant simplifications from reality). 

\subsection{Further work}
As shown in the attached Gantt chart (Appendix~\ref{sec:Gantt}), having completed much of the research into the background and context to the models, the research will proceed to fulfill the aims outlined in Section~\ref{sec:aims}. This mainly consists of building the models for the urban development design case and the water system operation case. Further research is required to ascertain how long-term stochastic waste management modelling can be integrated into the integrated model framework. The work will be then be brought together by assessing how well the models improve the metabolism of an area in order to minimise resource use and hence minimise the environmental strain and maximise the economic stability of an area. The immediate next steps are to expand the pump quality model to consider other processes in a water system (such as treatment and hydroelectric generation); and then integrate this water network into the Tat model, incorporating quality characteristics in other resources (such as biomass).

It is anticipated that three papers will be submitted during the course of the study (marked on the Gantt chart). These will correspond to each of the models. For example, the first paper could argue for the need for an integrated resource model which can incorporate resource quality characteristics and then describe the method to achieve this. The second paper can show how the method can be applied to optimise the design of a real-life urban development. A third paper will show how the methodology can be applied in an operation case study. Other potential material for publication could include a contribution to the urban metabolism literature regarding how this methodology can improve the urban metabolism of an area.

\subsection{Thesis outline}
The following is suggested as a chapter outline for the final thesis submission.
\begin{enumerate}
	\item \emph{Introduction---the urban context: trends, challenges and opportunities} \\
		This will assess global urbanisation trends and highlight the environmental and economic challenges relating to resource overconsumption associated with urban areas. Finally it will describe the opportunities urban areas provide for resource integration.
	\item \emph{Literature review} \\
		\begin{enumerate}
			\item \emph{Urban metabolism: a measure of resource circularisation} \\
				This will explain the urban metabolism concept and how it can be used to measure the resource efficiency of an area.
			\item \emph{Methodological background: a review of existing modelling methods in the energy, water and waste sectors} \\
				This review will show which techniques can be combined from these fields in order that an integrated model can be developed.
		\end{enumerate}
	\item \emph{Development of an integrated resource management model} \\
		This will explain the principles and the mathematics of the integrated resource management model.
	
	\item \emph{A prototype model for a village} \\
		The prototype model will provide a simple, easy to understand example.
	\item \emph{A model for urban design with spatial disaggregation} \\
		This will apply to a district of a city, and will be useful for design and urban planning purposes.
	\item \emph{A model for operation, with spatial and temporal disaggregation} \\
		This will be a water-based model (assuming good data can be found) to show how the modelling framework applies to the optimised operation of a system. For this reason, it will require temporal disaggregation.
	\item \emph{Conclusions and further work}
\end{enumerate}

