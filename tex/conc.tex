\section{Conclusions and further work}
\label{sec:conc}

\subsection{Conclusions}
Global urban trends have been examined and it has been found that the world is rapidly urbanising---especially in developing countries. Furthermore, it has been suggested that this urbanisation brings economic opportunity to the residents of towns and cities, thus any approach that seeks to mitigate against the environmental and economic challenges of urban areas must do so without opposing urbanisation itself. It is widely viewed that improved resource efficiency through integrated resource management is the best way to go about tackling these problems. Despite common agreement here, there is surprisingly little in the way of tools for optimising the integrated provision of energy, water and waste resources, even though they exhibit numerous interconnections and relationships within the urban environment. A review of optimised provision models within these fields has been undertaken with a view to developing an integrated resource model in this study. The first steps in developing a model have optimised resource flows in a small village (albeit with significant simplifications from reality). 

\subsection{Further work}
The following tasks are suggested for further work (see attached Gantt chart for details of timings).
\begin{itemize}
	\item The pump quality model for Tat needs to be expanded to consider the whole water network in Tat. Once a working water network has been developed, it can be integrated into the Tat model, along with quality considerations for other resources, such as biomass. 
	\item Further research needs to be done to see ascertain how long term, stochastic waste management models may be integrated into the waste management mode.
	\item Following this, more sophisticated resource integration models can be developed which take into account temporal and spatial disaggregation.
\end{itemize}

\subsubsection*{Thesis outline}
The following is suggested as a chapter outline for the final submission.
\begin{enumerate}
	\item \emph{Introduction---the urban context: trends, challenges and opportunities} \\
		This will assess global urbanisation trends and highlight the environmental and economic challenges relating to resource overconsumption associated with urban areas. Finally it will describe the opportunities urban areas provide for resource integration.
	\item \emph{Literature review} \\
		\begin{enumerate}
			\item \emph{Urban metabolism: a measure of resource circularisation} \\
				This will explain the urban metabolism concept and how it can be used to measure the resource efficiency of an area.
			\item \emph{Methodological background: a review of existing modelling methods in the energy, water and waste sectors} \\
				This review will show which techniques can be combined from these fields in order that an integrated model can be developed.
		\end{enumerate}
	\item \emph{Development of an integrated resource management model} \\
		This will explain the principles and the mathematics of the integrated resource management model.
	
	\item \emph{A prototype model for a village} \\
		The prototype model will provide a simple, easy to understand example.
	\item \emph{A model for urban design with spatial disaggregation} \\
		This will apply to a district of a city, and will be useful for design and urban planning purposes.
	\item \emph{A model for operation, with spatial and temporal disaggregation} \\
		This will be a water-based model (assuming good data can be found) to show how the modelling framework applies to the optimised operation of a system. For this reason, it will require temporal disaggregation.
	\item \emph{Conclusions and further work}
\end{enumerate}

