The following research question is proposed:
\begin{quote}
	\emph{By how much can the metabolism of an urban area be improved by creating and implementing models which optimise the integrated provision of energy, water and waste?}
\end{quote}
This question comprises three aims:
\begin{enumerate}
	\item Research the motivation, opportunity and methods for such models for the literature review. This will bring together up-to-date information on urban trends; a review of urban resource interactions and infrastrucures; and knowledge from existing methods used in optimising energy, water and waste management.
	\item Develop models to calculate the optimal transfer of resources through a network of processes such that demand for resources of required quality is met. The models must capture the complexity of considering multiple resources and their qualities whilst remaining tractable. Three models will be developed during the term of study:
		\begin{enumerate}[(i)]
			\item A prototype model based on a small subsistence-based community. There will be no spatial or temporal disaggregation in this model, because its primary purpose is to develop a methodology which can handle multiple resource types, each of which are associated with one or more quality parameters.
			\item A model to optimise resource management in an urban development (likely to be part of a city in China). This will introduce spatial disaggregation into the model.
			\item A water-based model to optimise the management of the energy-water nexus. Unlike the previous two models (which are focused on design), this model will consider system operation, and will therefore include temporal disaggregation.
		\end{enumerate}
	\item Asses how well the models improve urban metabolism through highlighting technological opportunities and pathways, and as a consequence, how well they minimise environmental impacts and ensure economic stability for urban areas.
\end{enumerate}

There are three (potentially) novel aspects to this research. Firstly, whilst there are tools which consider the optimisation of two resources (for example, the minimisation of energy requirements for meeting water demands in a town), there are no such tools which simultaneously optimise energy, water and waste management. Secondly, a new method will be required which incorporates resource quality requirements in a manner that makes the model computationally efficient and tractable. Finally, the application of optimisation methods to the field of urban metabolism is new ground, with the literature dominated by accounting studies.

