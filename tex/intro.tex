\section{Introduction}
\pagenumbering{arabic}
Introductory sentences.

The subject of this thesis is an interesting topic of study. It requires both a broad understanding of the big picture of urban resource management in three areas (energy, water and waste), including knowledge of policy environments etc. coupled with an understanding of mathematical programming for optimsation models.

The final outcome of this study will be a suite of tools which can be used to optimise the provision of water, waste and energy resources within an urban area, at a minimal financial and environmental cost, subject to various constraints. Before the methodology and models are described, the context of the project is outliend in this introduction, to motivate their development. Firstly, global urban trends are summarised before outlining the challenges and opportunities that come with urban areas. The concept of `urban metabolism' is introduced, which will provide the underlying theoretical framework for the study. Features of urban resource flows are then discussed in order to highlight opportunities for `resource integration'. Having established the context of the study, a precise `research question' is formulated to guide the rest of the research.

Who is this useful for: urban planners etc.

\subsection{Urban trends}
Three things can be said about urbanisation globally: (1) it is on the \emph{increase}; (2) this fact is \emph{important}; and (3) it follows from (1) and (2) that urbanisation is \emph{innevitable}.

Currently, urban areas are home to about half of the global population \citep{AREAS2012}\footnote{Note: statistics in this report will generally be from 2011 or earlier. More recent figures will not be recorded because they will need to be updated for the final submission anyway.}. There is a 
\citet{Cohen2006}
\citet{DepartmentforCommunitiesandLocalGovernment2006}
\citet{Audelo-Vera2011}
\citet{Cohen2004} Caution
\citet{York2011} Africa and asia

asdfa



Whilst urban areas serve a vital role to society, they are not without their challenges. These are now considered. 

\subsection{Challenges posed by urban areas}
There are two primary challenges posed by urban areas---those of \emph{environmental strain} and \emph{economic sustainability}.

However, as well as presenting these challenges, there are also unique opportunities inherent to urban spaces which can be exploited.

\subsection{Opportunities afforded by urban areas}
The density of urban areas, and the resulting co-location of infrastructures is the source of a potential solution to the challenges posed by urban areas.

An underlying theoretical framework is required which can help us measure the scale of the challenges presented; the opportunities on offer; and the success of the models. The 'urban metabolism' concept will serve this purpose.

Many studies suggest integrating infrastructures within their own field. This project is just an extension of that concept.

The definition of integration (literature may be different from this study).


\subsection{Urban metabolism---the theoretical framework}
Urban metabolism is the theoretical concept which supports this study in three ways:
\begin{enumerate}
	\item Urban metabolism studies give data which motivates the development of the models.
	\item It suggests am objective to achieve in the model: namely, a move away from resource `linearisation' towards `circularisation'.
	\item It provides a way of measuring the success of the models.
\end{enumerate}
Can be used to quantify the `challenges' discussed in that section.

\subsection{Aims and scope of the study}
Having now outlined the context and motivation for the study, a `research question' is posed. This gives the study a clear focus, and provides a way of testing the success of the final outcome of the study.
The following question will be answered:
\begin{quote}
	\emph{By how much can the metabolism of an urban area be improved by creating and implementing models which optimise the integrated provision of energy, water and waste?}
\end{quote}
This question comprises three aims:
\begin{enumerate}
	\item Research the motivation, opportunity and methods for such models for the literature review. This will bring together up-to-date information on urban trends; a review of urban resource interactions and infrastrucures; and knowledge from existing methods used in optimising energy, water and waste management.
	\item Develop models to calculate the optimal transfer of resources through a network of processes such that demand for resources of required quality is met. The models must capture the complexity of considering multiple resources and their qualities whilst remaining tractable. Three models will be developed during the term of study:
		\begin{enumerate}[(i)]
			\item A prototype model based on a small subsistence-based community. There will be no spatial or temporal disaggregation in this model, because its primary purpose is to develop a methodology which can handle multiple resource types, each of which are associated with one or more quality parameters.
			\item A model to optimise resource management in an urban development (likely to be part of a city in China). This will introduce spatial disaggregation into the model.
			\item A water-based model to optimise the management of the energy-water nexus. Unlike the previous two models (which are focused on design), this model will consider system operation, and will therefore include temporal disaggregation.
		\end{enumerate}
	\item Asses how well the models improve urban metabolism through highlighting technological opportunities and pathways, and as a consequence, how well they minimise environmental impacts and ensure economic stability for urban areas.
\end{enumerate}

There are three (potentially) novel aspects to this research: 
\begin{itemize}
	\item Whilst there are tools which consider the optimisation of two resources (for example, the minimisation of energy requirements for meeting water demands in a town), there are no such tools which simultaneously optimise energy, water and waste management. 
	\item Secondly, a new method will be required which incorporates resource quality requirements in a manner that makes the model computationally efficient and tractable. 
	\item Thirdly, the application of optimisation methods to the field of urban metabolism is new ground, with the literature dominated by accounting studies.
\end{itemize}



\subsection{Report structure}
Having presented the background to the study to provide motivation for the research question posed, the report continues by reviewing existing methods of optimising the provision of energy, water and waste in Section~\ref{sec:methods}. In Section~\ref{sec:models}, prototype models at their initial stage are presented. The report conludes in Section~\ref{sec:conc} with preliminary results, and an outline of the work to be undertaken in the rest of the project.

In summary, it has been argued cities need to focus on resource integration in order to reduce damage to the environment and economic risk. Currently, there are no models which consider the optimal management of multiple urban resources.
