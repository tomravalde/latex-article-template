\section{Introduction}
\pagenumbering{arabic}
Introductory sentences.

The rapid urbanisation of planet Earth is presenting both challenges and opportunitites in the way that finite resources are used in light of the challenges of climate change and resource depletion. \ldots This project therefore seeks to firstly take a broad look at the global picture of urbanisation and the challenges it presents, the policy background which arises out of this state of affairs and the ways in which the opportunities offer solutions to the challenges. Having considered the `big picture', the project focuses in specifically on the development of mathematical models  which optimise the integrated provision and management of energy, water and waste resources in an urban area. This Early Stage Assessment report therefore has two main aspects which reflect the ground to be covered in this study. Firstly, the big picture is considered as global urban trends are outlined and their implications as challenges and opportunities. The `urban metabolism' concept is introduced as an underlying theoretical concept which can be used to describe the links between trends, challenges and opportunities. This background information then motivates a research question. Secondly, the report moves on to describe various optimisation models in existence in the fields of energy, water and waste which can be used as a starting point in the development of an integrated resource management model, before going on to show the first steps already taken in developing a model.

The subject of this thesis is an interesting topic of study. It requires both a broad understanding of the big picture of urban resource management in three areas (energy, water and waste), including knowledge of policy environments etc. coupled with an understanding of mathematical programming for optimsation models.

The final outcome of this study will be a suite of tools which can be used to optimise the provision of water, waste and energy resources within an urban area, at a minimal financial and environmental cost, subject to various constraints. Before the methodology and models are described, the context of the project is outliend in this introduction, to motivate their development. Firstly, global urban trends are summarised before outlining the challenges and opportunities that come with urban areas. The concept of `urban metabolism' is introduced, which will provide the underlying theoretical framework for the study. Features of urban resource flows are then discussed in order to highlight opportunities for `resource integration'. Having established the context of the study, a precise `research question' is formulated to guide the rest of the research.

Who is this useful for: urban planners etc.

\subsection{The world is urbanising}
Three things can be said about urbanisation globally: (1) it is on the \emph{increase}; (2) this fact is \emph{important}; and (3) it follows from (1) and (2) that urbanisation is \emph{innevitable}. Firstly, the \emph{increase} in the global urban population is considered.

Currently, urban areas are home to about half of the global population \citep{AREAS2012}\footnote{Note: statistics in this report will generally be from 2011 or earlier. More recent figures will not be recorded because they will need to be updated for the final submission anyway.}. By 2030, it is expected that the urban population will rise to 61\% of the total, and again to 70\% by 2050, as cities absorb population growth, and continue to receive migrants from rural areas, aand rural areas become reclassified as urban \citep{Cohen2006}. Two notable features of this urbanisation are that firstly, most of the growth is concentrated in the developing world, with Africa and Asia together accounting for 86\% of the increase in urban dwellers by 2050 \citep{York2011}; and that despite there being much attention on `megacities', the majority of urban growth will be in smaller cities (below 500000 people) in the forseable future). Cities of more than 10 million residents will accomodate less than 10\% of the urban population \citep{Cohen2006}.

These forecasts aren't just the view of a small number, but widely accepted opinion. Thus even \citet{Cohen2004} who takes a skeptical look at urbanisation forecasts (pointing out that historical forecasts can be overestimates by more than 75\%; and that there is no universally consistent definition of `urban'\footnote{See for example \citet{DepartmentforCommunitiesandLocalGovernment2006}.} concludes: \emph{``Neverthelss, despite all the problems of error and inaccuracy and the long-standing definitional problems that have never been overcome, it is clear that the world is still in the midst of a sweeping and profound urban transformation that is literally changing the face of the planet.''}

It is natural to ask why such trends in urbanisation are seen, and whether this is good or bad. A useful description of the mechanism of urbanisation has been given by \citet{WorldBank2008a}. Urban centres arise as people migrate from the countryside for employment and better access to services (such as education, healthcare and employment). As cities grow, they beging to specialise with firms taking advantages of economies of scale. Cities become `agglomeration economies' when multiple types of firms and industries occupy the space, facilitating knowledge spillovers, and serving each other, such that individual firms can do their job better. At the same time, transport (and hence trade) costs reduce. Thus scale economies and agglomeration economies act to increase trade, and hence the prosperity of an area, attracting more migrants to the area. Thus a virtuous circle arises whereby city size and economic growth feed each other. This principle is seen in practise, given that one source estimates that 80\% of global GDP is generated in urban areas \citep{AREAS2012}.

This isn't to say that cities are without problems: the widespread presence of slums being one obvious concern (the World Bank describes these as `growing pains', and notes that they were historically seen in cities like London, but are no longer in existence there). However, in general, urbanisation is a clear route to economic prosperity. It is good news that the urbanisation patterns seen historically in OEDC countries are being replicated in the developing world \citep{WorldBank2008a}. This leads to the conclusion that urbanisation in \emph{important}.

Finally then, it must be the case that urbanisation is \emph{innevitable} due to the economic prosperity it brings to the developing world. Thus it is troubling to learn that \emph{``\ldots 72\% of developing countries have adopted policies designed to stem the tide of migration to their cities.''} GUARDIAN REFERENCE.

\subsection{Urbanisation poses challenges}
Having considered urbanisation trends and their benefits, the challenges of urbanisation are now considered. There are two primary challenges---those of \emph{environmental strain} and \emph{economic sustainability}.

\subsubsection{Environmental strain}
Considering the environmental strain, there are multiple problems associated with urban areas including land-use impacts, resource scarcity, poor water quality, emissions, the heat-island effect and eutrophicaion \citep{Cai2011}. A society starts to over-consume resources when the area (hinterlands) which serve it can't provide enough food, water and and other materials to meeds the needs of the inhabitants. A cycle ensues when an increasing population is required to managee the land to extract resources for that population. Thus even agriculturial driven societies (as well as industrial driven societies) can face resource scarcity \citep{Haberl2001a, Haberl2001b, Gr2003}. Even now individual cities are able to overcome the pressures on resources by looking beyond the immediate hinterlands, and making use of the transport system to be dependent on transport links, this is just spreading and growing the global environmental impact cities are responsible for \citep{Agudelo-Vera2011}. Thus, whilst currently home to just over 50\% of the global population, urban areas are responsible for around 60\% of primary energy use and 71\% of the global energy-related greenhouse gas emissions \citep{IEA2008}. For example, \citet{Grubler2009} shows the relationship between urban sprawl and the emissions from the associated increase in transport use (as journeys are required to be more frequent and longer).

Urbanisation results in a growing requirement for urban water delivery in view of growth in the urban population and a demand for a higher quality of life resulting in `water stress'. Associated with the increasing demand is the need to treat the water and prevent contamination. This demand for water in high quantitiies comes with a financial burden \citep{Diagger2009}. \citet{Kennedy} illusrates the problem of urban water sustainability by considering the relationship between urbanisation and groundwater. Shallow wells are sufficient to supply a population in the urban development stage, with wastewater discharging to a watercourse. As the population grows and water extraction rates rise, the water table falls requiring the digging of deeper wells and any shallow groundwater can become contminated by discharge wastewater. Thus water must be sourced (at significant expense) from outside the city, changing the city's groundwater characteristics such that the water table can rise above its original level and even cause flooding. In summary then, water security issues in urban areas are causing economic, health and environmental concerns.

Moving to consider waste management, there are a number of issues that urban areas face. For example, the risk of contaminating groundwater and public opposition to waste disposal near residential areas \citep{Li2006}, as well as growing waste generation rates and decreasing disposal capacity \citep{Lu2009}. In Greece, for example, municipal solid waste management has reached a `critical point'.

Moreover, the challenges faced in the energy, water and waste field are interalated. Population growth increases demands for water and energy significantly, and economic growth increases this demand on a per capita basis as people improve their living standards. With increased water demands, there will come increasing energy demands (for treatment and distribution) and increased wastewater generation. Furthermore, climate change will limit the availability of water in many places and will therefore induce high levels of spending and energy consumption in extracting water resources for human use (through desalination and other treatment methods). The increase in energy consumption will then exacerbate climate change invoking a self-reinforcing challenge \citet{Webber2011}.

There is now an increasing emphasis on developing policies to tackle the issues thrown-up by urbanisation. For example, \citet{Agudelo-Vera2011} argues that the objective of urban planning should include sustainable development (pointing out that urban planning has always sought to solve problems, such is accomodating larger populations and meeting transport requirements). It then points out that resource management is an essential part of sustainable development, and that resource management shouldn't remain in the past where it's main aim was to meet demand. Thus, resource management needs to be integrated into urban planning in order to end the pattern of over consumption and excessive waste production which the global eco-system cannot carry. A specific policy example is the European Union target of 20\% of final energy to be provided by renewable sources by 2020 \citep{Keirstead2012}.

London provides a good example of a city experiencing the problems and therefore developing policies described in this section. For example, London has a comprehensive range of policies concerning transport, the retrofitting of housing and the decentralisation of energy supply in order to reduce energy-related emissions \citep{Davis2011}. In the water sector, London is under great stress, suffering from an old network which leaks about 25\% of the water which enters it. When this is coupled with an increasing supply (as population grows, with a tendency to live in houses of fewer people) and the reduced availability of water due to climate change, then it is obvious that current consumption (already above the national average) is unsustainable. Thus policies are being implemented to capture rainwater and wastewater as a resource; change consumer behaviour and payment patterns; and make buildings more efficient \citep{Nickson2011}. Finally, considering London's municipal waste, 49\% currently ends up in landfill, some of this outside London; policies are being implemented to build new waste management infrastructure, increase recylcling and use waste incineration as a method of energy production, for example with the SELCHP incinerator \citep{Zabal2011}.
%\citep{Chen2006} Link to resource use, environment, policy and technology options.

%\citep{Batt2010} to summarise.
In summary, this study is in agreement with \citet{Newman1999} that the sustainability of cities needs to be improved through better resource management. It must be remembered however, that a city isn't simply a resource proceeing machine, but it is the home to many people, and the place where they seek economic opportunity.  Therefore, there is a particular challenge to better manage resources without harming the `livability' of cities. Another conclusion that is innevitable from the survey of challanges posed by cities is that resource efficiency isn't enough. The world is already struggling to sustain the global population, so a `per capita' decrease in consumption in a growing population will not be enough on it's own (for example, at current trends, the entire global biomass would be required to meet energy needs by 2050). There must be dematerialisation in absolute terms, especially when a more efficient use of resources may cause people to seek a higher standard of living---the so called `take-back' effect \citep{Winiwater2011}.

\subsubsection{Economic stability}
Moving to consider the economic sustainability of cities. This section is largely based on the work of \citet{Bettencourt2007} who seeks a quantitative understanding of \emph{``human social organization and dynamics in cities''}. As entities that both consume and produce, they are analagous to living organisams, who's properties scale in a self-similar manner with mass. In the case of a city, indicators, $Y$ (such as GDP or energy consumption) scale with population, $N$ (`Zipf's law) as:
\begin{align} \label{eq:urban_scale}
	Y(t)=Y_0N(t)^{\beta}
\end{align}
where $t$ is time, $Y_0$ is a normalisation constant and $\beta$ is a constant which corresponds to the indicator under consideration. It is observed that $\beta<1$ for quantities related to materials or infrastructure (such as the length of electrical cables), thus exhibiting economies of scale as population growth. However, for innovation driven indicators such as income, employment in R\&D and patents registered, $\beta>1$, thus the pace of life with respect to these indicators increases as population grows. %Thus material economies of scale is in tension with the super-linear scaling of innovation characteristics. 

If the growth of city is constrained by the availability of resources such that $R$ resource are required per unit time to maintain a member of the existing population, and $E$ resources are required to add a new member, urban growth can be described as:
\begin{align} \label{eq:urban_growth}
	Y=RN(t)+E\frac{dN(t)}{dt}
\end{align}
Equating Equations~\eqref{eq:urban_scale} and~\eqref{eq:urban_growth}, and rearranging, urban growth is modelled as:
\begin{align}
	\frac{dN(t)}{dt}=\left(\frac{Y_0}{E}\right)N(t)^{\beta}-\left(\frac{R}{E}\right)N(t),
\end{align}
which has the solution:
\begin{align}
	N(t)=\left\{\frac{Y_0}{R}+\left(N^{1-\beta}(0)-\frac{Y_0}{R}\right)\exp\left[-\frac{R}{E}(1-\beta)t\right]\right\}^{\frac{1}{1-\beta}}
\end{align}

In an `agglomeration economy', which is driven by knowledge and innovation ($\beta>1$), $N$ will grow with $t$ at a greater-than-exponential rate, theoretically reaching an infinite population in a finite time, $t_c$:
\begin{align}
	t_c&=-\frac{E}{(\beta-1)R}\ln\left[1-\frac{R}{Y_0}N^{1-\beta}(0)\right] \\
	&\approx \left[\frac{E}{(\beta-1)R}\right]\frac{1}{N^{\beta-1}(0)}.
\end{align}
However, since resources are limited, there will come a time when they run out, and stagnation and collapse occurs. Such collapse can be avoided if a transformation is undergone which resets the initial conditions $Y_0$ and $N(0)$, to begin a new cycle of greater-than-exponential growth. However, $t_c$ for each cycle becomes shorter and shorter requiring innovation to take place at increasingly faster paces. Thus, efficient resource management is a key challenge in urban design, because this effectively increases the value of $R$ and hence buys innovation time to push the stagnation point into the future.
% PLOT GRAPHS AT A LATER DATE

In conclusion, there are two major challenges that urban planning must face in the future: firstly that of using resources more efficiently in order to minimise environmental strain caused by cities and secondly that of avoiding economic collapse as cities outgrow their resource feed. In both cases, the root cause is an over-consumption of resources. 

\subsection{Urban areas present opportunities}
Having seen the scale of the environmental and economic challenges posed by cities, this section will demonstrate that urban areas are home to great opportunity to tackle the root problem of excessive resource consumption. This is provided due to the density of urban areas, and the resulting co-location of infrastructures. Thus, \emph{``urban planning in the 21st Century should evolve towards an ecologically-oriented macro-architecture fully integrating the design and location of energy- and material-efficient buildings and urban infrastructure with overall spatial planning further to minimize material throughput.''} REESREFERENCE. The dense nature of cities is such that synergies between urban infrastructures can be identified and exploited. Thus a node in a resourece management network can have multiple functionalities and the waste products of one process can become the inputs to another. Examples include waste-to-energy, biogas generation from anaerobic digestion of sewage sludge, hydropower and the treatment and redistribution of wastewater \citep{Kharrazi2012}. This is what \citet{Leduc2010} refers to as the `urban harvest' approach, which takes advantage of connectivity and proximity in cities to maximise the resuse of water, waste and other materials within an urban system, and the cascading of energy through multiple uses in order to `close' cycles. An example of energy cascading would be the cogeneration of electricity and heat \citep{Grubler2009}. Such `circularisation' of resources is increasingly playing a part in sustainable urban design \citep{Meijer2011}. The rest of this section goes on to outline some of the specific resource interactions which exist in urban environments.
% General
%The concepts that will be advocated include:
%- systems perspective
%- integration
%- decentralisation
%- closing the loop
%- cascading
%- synergies
%\citep{Cai2011} Systematic and intensive municiple infrastructure e.g. integrated water systems
%\citep{Clift2000} Systems approach

% Energy
%% Point out that there are inherent advantages to DER: price, reduces emissions, developing countires, fuel prices, low carbon construction
%% Specific resource integration advantages: local management, and local resources.
Firstly, current thinking and opportunities in the urban energy system will be  reviewed. This will provide the starting point for considering the energy-water and energy-waste intersections in subsequent paragraphs. The electricity sector is seeing increasing favouring `distributed' or `decentralised' provision, since this reduces power losses (in electricity networks), reducing emissions and electricity costs \citep{Fleten2007}. What makes distributed generation of specific interest to this study is the fact that these systems incorporate a \emph{``wide range of technologies in cluding combined heat and power plant (CHP), photovoltaic systems (PV), small wind turbines and other systems using renewable energy sources (e.g. biogas digesters)''} with the intention that local resources can be used (for example biomass) \citep[p. 1001]{Ren2010}. Both the location of the systems in urban centres (at multiple locations) and the use of local resources facilitates their utility in taking advantage of resource intersections, infrastructure synergies and integrated systems. Models already confirm the feasibility and environmental and financial benefits of integrating distributed systems with centralised systems, and district heating networks in order to meet the energy demands of a city \citep{Weber2011}. Another (yet related) move in energy systems is to take advantage of energy `cascading' (as well as integration), for example in CHP, whereby waste heat in electricity production is used in heating systems \citep{Grubler2009}. In energy markets, this will correspond as companies start to provide `integrated energy services' (selling both heat and electricity to the consumer) \citep{Sugihara2004}. Biomass is a resource that can similarly be `cascaded' in ``energetic recycling" using residues and wastes in energy provision, rather than appropriating more natural resources for the same person \citep{Haberl2001a}.
%%\citep{Kierstead2012b} Argues for integration (4.3)

% Water-energy-wastewater nexus
Having considered the shape of the energy industry specifically, this report now moves on to consider the possible interactions and intersections the the urban environment gives rise to, starting with the energy-water nexus\footnote{This discussion will include the place of wastewaster, thus it could be considered relevent to the waste-water nexus, which won't be considered separately at this point}. An extensive survey of both the energetic needs of water managment and the opportunities for energy generation by water provided by \citep{McMahon2011} concludes there is a high level of interdepenendence between water and electricity which will increase in the future as global energy consumption is expected to increase by 49\% between 2007 and 2035. To meet the challenge of the extensive water use required in cooling (in the electricity industry) and the use of energy in the pumping, treatment and end-uses of water, the authors advocate `closed-loop cooling' (which could be describes as `water cascading', water re-use and reclamation (to minimise raw water use at the expense of increased electricity requirements), co-location of water and energy infrastructure. \citet{Daigger2009} sees the answer to the problems of water stress in a toolkit of \emph{``closed-loop urban water and resource management systems''}, which includes: rainwater harvesting, conservation, reclamation and reuse, energy management, nutrient and source separation managed in centralised and decentralised forms, in order to move away from the linear \emph{``take, make, waste''} approach. For example, roof collectors and permeable are decentralised methods of capturing rainwater; houses can contain systems to separate kitchen water and grey water locally, for separate treatments and uses. At the centralised level, anaerobic digestion of waste-water can be carried out converts the organic content of wastewater to biogas (which can then be used in CHP applications). The energy requirements of water are also reduced in systems where reclaimed water is treated (compared to the treatment and distribution in ordinary water supply). Nutrient recovery from wastewater saves energy through displacing it from  energy-intensive processes such as fertilizer production. Source separation separate various streams of water for various uses---for example a potable/non-potable division means that water for use in toilet flushing requires less treatment than drinking water, resulting in the provision of additional water supplies to meet needs. Kitchen and toilet water, with its high organic content can go towards energy production, through decentralised multiple pipe networks. \citet{Makropoulos2008} categorises urban water as potable (for drinking), grey (dilute wastewater which can be reused in toilet flushing or garden irrigation), greenwater (treated rain water and wastewater, for domestic use), wastewater (requiring treatment before discharge into the environment) and runoff. It is argued that an integrated water system needs to understand how these systems interact (such as the effect of recylcing on treatment) in ordeer to optimise an urban water system. When more than one of these options is combined in urban water management, the result is an `integrated system' such that net water consumption reduces and energy performance improves. \citet{Hardy2005} develops a model to optimise such an integrated water resource network, using `systems framework'. Such opportunties in the urban water system exist at multiple scales (for example the household scale and a community scale) \citep{Makropoulos2008}.
%% INCLUDE FIG 5 and 6. 
%%challenge of bioenergy increase in water use. (McMahon).
%%\citep{Lim2010} Heirarchical water allocation, integration of infrastructure, reducing electricity consumption from pumping etc.
%% footnote on how wastewater is classified is required.
%%\citep{Makropulos2008} Tool for IWM based on an holistic view. Tradeoffs
%%\citep{Nickson2011} Rainwater, waste-water (for energy) and reclaimed water (reduces pumping).

% Waste-energy nexus
The waste-energy nexus is also receiving a lot of attention as another opportunity for urban resource integration. \citet{Geng2010} reuse, recycling and recovery are becoming more popular. Incineration remains a popular method of waste management because it both saves space and can generate power of heat. Despite  this, there is opposition to new plant in local areas, so it isn't achieving the gains it should be. `Industrial symbiosis' is advocated whereby firms use each others' waste outputs as raw material inputs in other processes (e.g. production and manufacturing as well as waste-to-energy applications), inline with the `proximity principle' observed in Japan, whereby waste is managed close to source. For example, the OWARE model of \citep{Eriksson2002} takes a systems perspective on all the possible sources and uses of waste in a system (such as a city) by combining various sub-models which consider the life-cycle costs associated with incinceration, landfill, recyling , anaerobic digestion and so on.
Another specific waste-energy intersection is discussed by \citet{Iacovidou2012}: that of the anaerobic co-digestion of food waste with sewage sludge (also intersecting with the water sector) as a source of renewable energy, reducing landfill (and creating fertiliser). Arising from this would be a economic benefits from reduced infrastructure costs. A simple model to optimise the flows in an energy-waste network is proposed by \citep{Kharrazi2012} as an example of an integrated resource network taking inspiration from ecological principles. Nodes represent the producers and consumers of waste and electricity, and links between the nodes are the flows of those resources. The sustainability of the network can assessed using `ecological network analysis' which quantifies the efficiency and resiliance of the network. A real world application of waste-to-energy networks is seen in Greece. The shortage of landfill can be dealt with by the increase of WTE unit peneration which will support existing electricity generation systems \citep{Xydis2012}.

% Water-waste nexus
\citep{Lundin2000} Evidence that recycling waste-water makes emissions savings and provides useful outputs.


An underlying theoretical framework is required which can help us measure the scale of the challenges presented; the opportunities on offer; and the success of the models. The 'urban metabolism' concept will serve this purpose.

Many studies suggest integrating infrastructures within their own field. This project is just an extension of that concept.

The definition of integration (literature may be different from this study).
The term `integration' can have multiple meanings

Furthermore, there is little research in this field (despite many calls for it).  

\subsection{Urban metabolism---the theoretical framework}
Urban metabolism (UM) is the theoretical concept which undergirds this study. For example, the discussion surrounding the linearisation and circularisation of resources is actually a comment on the nature of the metabolism of an environment. 
Conducting a study of an area's metabolism can be used to understand the processes within it. Such studies imagine the city as an ecosystem and aim to quanity the fluxes of water, energy, waste and other materials into and out of urban populations in order to find the \emph{``sum total of the technical and socioeconoic processes that occur in cities, resulting in growth, production of energy and elimination of waste''} \citep[p. 44]{Kennedy2008}. Invidual studies vary in their scope, about the inclusion of energy, water, material and nutrient components. Assessing these quantities will give an indication of the sustainability of a city. The basic methdology in an UM study is a `material flow analysis' (MFA) where resource flows in and out of areas are measured to see where they accumulate \citep{Barles2009}. Inputs included `domestic extraction' (DE) or `imports' whilst outpus are `exports', `deliberate disposal' (DD) or `wastes and emissions' (WE). Operations on these variables give `indicators' which can be used to compare the metabolism of different locations (or the same location at different times) to assess the relative sustainability of an area\footnote{For example, $\mbox{`material intensity'}=\mbox{DMI}/\mbox{GDP}$. This value reduces as a society dematerialises.} \citep{Hobbes2005}. \citet{Haberl2001a, Haberl2001b} show that thee indicator `human appropriation of net primary production' (HANNP), `domestic sonsumption' and `total energy input' divides societies into three categories: hunter-gather, agricultural and industrial. Each of these exhibits qualitatively different metaboliic profiles. Hunter-gather communities have no struggle meeting there needs from their environment, however agricultural communities will reach a `labour' limit where the environment will not be able to meet the needs of those working on it, because their energetic needs (including that embodied in food and materials) simply cannot be met by the `biological productivity' of the area---thus even agricultutral regimes can be unsustainable. Having defined UM, it can be seen that it offers a way to quantify the types of urban sustainability challenges presented by the over-consumption of resources. \citet{Barles2010} makes the case for the importance of urban metabolism studies in sustainable urban development issues, noting how research has moved on from viewing cities as `parasites', feeding off the environment at its expense; but rather the UM approach has allowed the consideration of optimisation, making use of possible industrial symbioses.

\citet{Kennedy2008} demonstrates the power of UM studies in a review of studies of eight regions around the world since 1965. The authors conclude that in general, there is an increasing per capita metabolism for water, wastewater, energy and materials. Nevertheless, Toronto has seen per capita reductions in energy and water use through increased efficiency, and increased recyling rates have reduced the per capita use of waste in multiple locations. Usefully, \citet{Kennedy2008} lists water, energy and materials (which end up as waste) as the largest components in UM studies. Thus justifying the approach of this study in targetting these resources for optimised resource integration opportunities---a result also seen for London in \citet{BFF2002}. An UM study can identify specific areas for dematerilisation of a region. For example, a multi-scale UM study of Paris and the surrounding region in \citet{Barles2009} shows the city's dependence on a large area for its waste management. Thus, actions should be targeted in that secotr to improve Paris's sustainability.

A particular challenge noted in the literature is the coupling between DMC, DMI and GDP, for example, in the case of Singapore \citep{Shulz2007}. Thus \citet{Kowalski2013} argues challenge that sustainable design is facing is the need to de-link prosperity and quality of life from metabolism---currently these notions feed each other (this is a similar case to that made by the World Bank concerning the development of aggolmeration economies). This can be achieved as technologies increase in efficiency faster than the economic growth of a society (aswell as brining about changes in cultural expectations anf behaviours). The way that such efficiency improvements can be achieved at the urban scale (as opposed to the individual technology scale) is through the circularisation of urban resource flows (rather than the current linear approach)---which ultimately reduces inputs, as argued by \citet{Newman1999} in the `extended metabolism model'. This approach wants to recognise that a city is more than a mechanism for the physical processing of resources, but also the centre of human opportunity. 

Thus, now having an appreciation for the utility of UM studies in assessing the resource efficiency of an urban area, it can be seen why \citet{Kennedy2011} advocates in urban design, using MFAs to produce \emph{``more ecologically sensetive designs''} for a city through infrastructure integration, closing loops to minimise overall inputs and outpus of energy, water and materials. In conclusion then, it can be said that the urban metabolism concept is key to improving resource management in order to bring sustainable design considerations to urban planning \citep{Agudelo-Vera2011}.
%% definitions
%% challenges
% \citep{Chen2012} Measure of metabolism given by network environ theory. Indicators for linkage and synergism.
%% Opportunities
%% Policy and planning
% \citep{Hobbes2007} social context to relate UM to actors, decisions. Explanatory approach to MFA.

Urban metabolism is useful in three ways:
\begin{enumerate}
	\item Urban metabolism studies give data which \emph{motivates} the development of the models. 
	\item It suggests the \emph{means} by which urban resource management can be improved: namely, a move away from resource `linearisation' towards `circularisation'. 
	\item It provides a way of \emph{measuring} the success of the models. 
\end{enumerate}
Can be used to quantify the `challenges' discussed in that section.


\subsection{Aims and scope of the study}
Having now outlined the context and motivation for the study, a `research question' is posed. This gives the study a clear focus, and provides a way of testing the success of the final outcome of the study.
The following question will be answered:
\begin{quote}
	\emph{By how much can the metabolism of an urban area be improved by creating and implementing models which optimise the integrated provision of energy, water and waste?}
\end{quote}
This question comprises three aims:
\begin{enumerate}
	\item Research the motivation, opportunity and methods for such models for the literature review. This will bring together up-to-date information on urban trends; a review of urban resource interactions and infrastrucures; and knowledge from existing methods used in optimising energy, water and waste management.
	\item Develop models to calculate the optimal transfer of resources through a network of processes such that demand for resources of required quality is met. The models must capture the complexity of considering multiple resources and their qualities whilst remaining tractable. Three models will be developed during the term of study:
		\begin{enumerate}[(i)]
			\item A prototype model based on a small subsistence-based community. There will be no spatial or temporal disaggregation in this model, because its primary purpose is to develop a methodology which can handle multiple resource types, each of which are associated with one or more quality parameters.
			\item A model to optimise resource management in an urban development (likely to be part of a city in China). This will introduce spatial disaggregation into the model.
			\item A water-based model to optimise the management of the energy-water nexus. Unlike the previous two models (which are focused on design), this model will consider system operation, and will therefore include temporal disaggregation.
		\end{enumerate}
	\item Asses how well the models improve urban metabolism through highlighting technological opportunities and pathways, and as a consequence, how well they minimise environmental impacts and ensure economic stability for urban areas.
\end{enumerate}

There are three (potentially) novel aspects to this research: 
\begin{itemize}
	\item Whilst there are tools which consider the optimisation of two resources (for example, the minimisation of energy requirements for meeting water demands in a town), there are no such tools which simultaneously optimise energy, water and waste management. 
	\item Secondly, a new method will be required which incorporates resource quality requirements in a manner that makes the model computationally efficient and tractable. 
	\item Thirdly, the application of optimisation methods to the field of urban metabolism is new ground, with the literature dominated by accounting studies.
\end{itemize}



\subsection{Report structure}
Having presented the background to the study to provide motivation for the research question posed, the report continues by reviewing existing methods of optimising the provision of energy, water and waste in Section~\ref{sec:methods}. In Section~\ref{sec:models}, prototype models at their initial stage are presented. The report conludes in Section~\ref{sec:conc} with preliminary results, and an outline of the work to be undertaken in the rest of the project.

In summary, it has been argued cities need to focus on resource integration in order to reduce damage to the environment and economic risk. Currently, there are no models which consider the optimal management of multiple urban resources.

\subsection{Summary}
In conclusion, there is strong case to be made that research into a model that optimises the provision of urban energy, water and waste resources should be persued. This is because urbanisation has socioeconomic advantages, such that it should not be resisted, despite the challenges it brings. Rather, urban centres provide an opportunity with the proximity and co-location of infrastructure to meet the challenges of resource management. On top of this there is a research-gap in the field of integrated resource provision.
